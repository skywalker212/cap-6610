% THIS TEMPLATE IS A WORK IN PROGRESS

\documentclass{article}

\usepackage{hyperref}
\usepackage{fancyhdr}

\fancypagestyle{firstpage}{%
  \lhead{CAP6610 Home Work 1 Report}
  \rhead{Akash Gajjar}
}

\begin{document}
\thispagestyle{firstpage}


\section*{Implementation}

For the solution I have used Python version 3.8.10 . For operations related to linear algebra and to generate samples from Gaussian distribution with various parameters I am using \emph{numpy}, to plot the graphs from results I am using \emph{matplotlib}. I have implemented multiple functions to generate data and to perform Regression or Ridge Regression on the data and calculate the L2 distance between $\hat{\Theta}$ and $\Theta^{*}$. The function to perform the regression is implemented in such a way that if $N<K$ then it will perform Ridge Regression otherwise it will perform Regression using OLS.

\section*{Graphs}

I have noticed that when $N<K$, sometimes we are not able to find the inverse of the matrix $X^{T}X$ or even if we are able to find the inverse of the matrix, the values are too high which results in L2 distance being very high. Below are 
\end{document}
